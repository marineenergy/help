% Options for packages loaded elsewhere
\PassOptionsToPackage{unicode}{hyperref}
\PassOptionsToPackage{hyphens}{url}
%
\documentclass[
]{book}
\title{A Minimal Book Example}
\author{Yihui Xie}
\date{2023-04-06}

\usepackage{amsmath,amssymb}
\usepackage{lmodern}
\usepackage{iftex}
\ifPDFTeX
  \usepackage[T1]{fontenc}
  \usepackage[utf8]{inputenc}
  \usepackage{textcomp} % provide euro and other symbols
\else % if luatex or xetex
  \usepackage{unicode-math}
  \defaultfontfeatures{Scale=MatchLowercase}
  \defaultfontfeatures[\rmfamily]{Ligatures=TeX,Scale=1}
\fi
% Use upquote if available, for straight quotes in verbatim environments
\IfFileExists{upquote.sty}{\usepackage{upquote}}{}
\IfFileExists{microtype.sty}{% use microtype if available
  \usepackage[]{microtype}
  \UseMicrotypeSet[protrusion]{basicmath} % disable protrusion for tt fonts
}{}
\makeatletter
\@ifundefined{KOMAClassName}{% if non-KOMA class
  \IfFileExists{parskip.sty}{%
    \usepackage{parskip}
  }{% else
    \setlength{\parindent}{0pt}
    \setlength{\parskip}{6pt plus 2pt minus 1pt}}
}{% if KOMA class
  \KOMAoptions{parskip=half}}
\makeatother
\usepackage{xcolor}
\IfFileExists{xurl.sty}{\usepackage{xurl}}{} % add URL line breaks if available
\IfFileExists{bookmark.sty}{\usepackage{bookmark}}{\usepackage{hyperref}}
\hypersetup{
  pdftitle={A Minimal Book Example},
  pdfauthor={Yihui Xie},
  hidelinks,
  pdfcreator={LaTeX via pandoc}}
\urlstyle{same} % disable monospaced font for URLs
\usepackage{color}
\usepackage{fancyvrb}
\newcommand{\VerbBar}{|}
\newcommand{\VERB}{\Verb[commandchars=\\\{\}]}
\DefineVerbatimEnvironment{Highlighting}{Verbatim}{commandchars=\\\{\}}
% Add ',fontsize=\small' for more characters per line
\usepackage{framed}
\definecolor{shadecolor}{RGB}{248,248,248}
\newenvironment{Shaded}{\begin{snugshade}}{\end{snugshade}}
\newcommand{\AlertTok}[1]{\textcolor[rgb]{0.94,0.16,0.16}{#1}}
\newcommand{\AnnotationTok}[1]{\textcolor[rgb]{0.56,0.35,0.01}{\textbf{\textit{#1}}}}
\newcommand{\AttributeTok}[1]{\textcolor[rgb]{0.77,0.63,0.00}{#1}}
\newcommand{\BaseNTok}[1]{\textcolor[rgb]{0.00,0.00,0.81}{#1}}
\newcommand{\BuiltInTok}[1]{#1}
\newcommand{\CharTok}[1]{\textcolor[rgb]{0.31,0.60,0.02}{#1}}
\newcommand{\CommentTok}[1]{\textcolor[rgb]{0.56,0.35,0.01}{\textit{#1}}}
\newcommand{\CommentVarTok}[1]{\textcolor[rgb]{0.56,0.35,0.01}{\textbf{\textit{#1}}}}
\newcommand{\ConstantTok}[1]{\textcolor[rgb]{0.00,0.00,0.00}{#1}}
\newcommand{\ControlFlowTok}[1]{\textcolor[rgb]{0.13,0.29,0.53}{\textbf{#1}}}
\newcommand{\DataTypeTok}[1]{\textcolor[rgb]{0.13,0.29,0.53}{#1}}
\newcommand{\DecValTok}[1]{\textcolor[rgb]{0.00,0.00,0.81}{#1}}
\newcommand{\DocumentationTok}[1]{\textcolor[rgb]{0.56,0.35,0.01}{\textbf{\textit{#1}}}}
\newcommand{\ErrorTok}[1]{\textcolor[rgb]{0.64,0.00,0.00}{\textbf{#1}}}
\newcommand{\ExtensionTok}[1]{#1}
\newcommand{\FloatTok}[1]{\textcolor[rgb]{0.00,0.00,0.81}{#1}}
\newcommand{\FunctionTok}[1]{\textcolor[rgb]{0.00,0.00,0.00}{#1}}
\newcommand{\ImportTok}[1]{#1}
\newcommand{\InformationTok}[1]{\textcolor[rgb]{0.56,0.35,0.01}{\textbf{\textit{#1}}}}
\newcommand{\KeywordTok}[1]{\textcolor[rgb]{0.13,0.29,0.53}{\textbf{#1}}}
\newcommand{\NormalTok}[1]{#1}
\newcommand{\OperatorTok}[1]{\textcolor[rgb]{0.81,0.36,0.00}{\textbf{#1}}}
\newcommand{\OtherTok}[1]{\textcolor[rgb]{0.56,0.35,0.01}{#1}}
\newcommand{\PreprocessorTok}[1]{\textcolor[rgb]{0.56,0.35,0.01}{\textit{#1}}}
\newcommand{\RegionMarkerTok}[1]{#1}
\newcommand{\SpecialCharTok}[1]{\textcolor[rgb]{0.00,0.00,0.00}{#1}}
\newcommand{\SpecialStringTok}[1]{\textcolor[rgb]{0.31,0.60,0.02}{#1}}
\newcommand{\StringTok}[1]{\textcolor[rgb]{0.31,0.60,0.02}{#1}}
\newcommand{\VariableTok}[1]{\textcolor[rgb]{0.00,0.00,0.00}{#1}}
\newcommand{\VerbatimStringTok}[1]{\textcolor[rgb]{0.31,0.60,0.02}{#1}}
\newcommand{\WarningTok}[1]{\textcolor[rgb]{0.56,0.35,0.01}{\textbf{\textit{#1}}}}
\usepackage{longtable,booktabs,array}
\usepackage{calc} % for calculating minipage widths
% Correct order of tables after \paragraph or \subparagraph
\usepackage{etoolbox}
\makeatletter
\patchcmd\longtable{\par}{\if@noskipsec\mbox{}\fi\par}{}{}
\makeatother
% Allow footnotes in longtable head/foot
\IfFileExists{footnotehyper.sty}{\usepackage{footnotehyper}}{\usepackage{footnote}}
\makesavenoteenv{longtable}
\usepackage{graphicx}
\makeatletter
\def\maxwidth{\ifdim\Gin@nat@width>\linewidth\linewidth\else\Gin@nat@width\fi}
\def\maxheight{\ifdim\Gin@nat@height>\textheight\textheight\else\Gin@nat@height\fi}
\makeatother
% Scale images if necessary, so that they will not overflow the page
% margins by default, and it is still possible to overwrite the defaults
% using explicit options in \includegraphics[width, height, ...]{}
\setkeys{Gin}{width=\maxwidth,height=\maxheight,keepaspectratio}
% Set default figure placement to htbp
\makeatletter
\def\fps@figure{htbp}
\makeatother
\setlength{\emergencystretch}{3em} % prevent overfull lines
\providecommand{\tightlist}{%
  \setlength{\itemsep}{0pt}\setlength{\parskip}{0pt}}
\setcounter{secnumdepth}{5}
\usepackage{booktabs}
\usepackage{amsthm}
\makeatletter
\def\thm@space@setup{%
  \thm@preskip=8pt plus 2pt minus 4pt
  \thm@postskip=\thm@preskip
}
\makeatother
\ifLuaTeX
  \usepackage{selnolig}  % disable illegal ligatures
\fi
\usepackage[]{natbib}
\bibliographystyle{apalike}

\begin{document}
\maketitle

{
\setcounter{tocdepth}{1}
\tableofcontents
}
\hypertarget{prerequisites}{%
\chapter{Prerequisites}\label{prerequisites}}

This is a \emph{sample} book written in \textbf{Markdown}. You can use anything that Pandoc's Markdown supports, e.g., a math equation \(a^2 + b^2 = c^2\).

The \textbf{bookdown} package can be installed from CRAN or Github:

\begin{Shaded}
\begin{Highlighting}[]
\FunctionTok{install.packages}\NormalTok{(}\StringTok{"bookdown"}\NormalTok{)}
\CommentTok{\# or the development version}
\CommentTok{\# devtools::install\_github("rstudio/bookdown")}
\end{Highlighting}
\end{Shaded}

Remember each Rmd file contains one and only one chapter, and a chapter is defined by the first-level heading \texttt{\#}.

To compile this example to PDF, you need XeLaTeX. You are recommended to install TinyTeX (which includes XeLaTeX): \url{https://yihui.name/tinytex/}.

\hypertarget{sysadmins}{%
\chapter{System Administrators}\label{sysadmins}}

\hypertarget{server-software}{%
\section{Server Software}\label{server-software}}

\begin{itemize}
\tightlist
\item
  \url{https://github.com/mhk-env/mhk-env_server-software}
\end{itemize}

Symbolically link (\texttt{ln\ -s}) a Shiny server app from within the Github repo (\texttt{mhk-env\_shiny-apps}) to the active folder \texttt{/srv/shiny-server} from where Shiny apps get served:

\begin{verbatim}
ln -s /share/github/mhk-env_shiny-apps/report-gen2 /srv/shiny-server/report
\end{verbatim}

\hypertarget{old-bookdown}{%
\section{OLD (bookdown)}\label{old-bookdown}}

This website describes how to use open-source software and data to construct the \href{https://MarineEnergy.app}{MarineEnergy.app} Toolkit for Enviromental Compliance, organized for now by audience.

\hypertarget{rtech}{%
\chapter{R Technicians}\label{rtech}}

\hypertarget{updating-dynamic-content}{%
\section{Updating dynamic content}\label{updating-dynamic-content}}

\begin{itemize}
\tightlist
\item
\end{itemize}

\hypertarget{shiny-app-content-updates}{%
\subsection{Shiny App Content Updates}\label{shiny-app-content-updates}}

To render changes made to content in a Shiny app, navigate to the folder that contains the application files to run the app. Shiny apps are served from a standard folder where they are symbolically linked to source files. These folders contain the global.R, ui.R, server.R scripts that run the app.

\begin{itemize}
\tightlist
\item
  global.R: functions used in ui.R and server.R can be defined here, sourced here from other files or in loaded libraries
\item
  ui.R: defines the user interface, or display, of the contents in the shiny app; this is navigable through the bookmarks in the lower left menu of the Code panel in RStudio
\item
  server.R: back-end server functions to populate the user interface; this is also navigable through the bookmarks in the lower left menu of the Code panel in RStudio
\end{itemize}

\hypertarget{projects-page}{%
\subsection{Projects Page}\label{projects-page}}

The projects page is accessed from the marineenergy.app Reporting Tool. Projects shown on the map and timeline diagram are updated periodically based on input from stakeholders. The content is updated through a series of steps as outlined below.

\begin{itemize}
\tightlist
\item
  Enter new project information into: \href{https://docs.google.com/spreadsheets/d/1MTlWQgBeV4eNbM2JXNXU3Y-_Y6QcOOfjWFyKWfdMIQM/edit?usp=sharing}{data \textbar{} marineenergy.app - Google Sheet}

  \begin{itemize}
  \tightlist
  \item
    If a new project permit type is needed, add this to the list presented on the project\_permit\_types worksheet
  \end{itemize}
\item
  Use RStudio to open the apps\_dev/scripts/update\_all.R script within the /share/github/apps\_dev repository branch

  \begin{itemize}
  \tightlist
  \item
    Run the update\_projects() function

    \begin{itemize}
    \tightlist
    \item
      Note that this function is defined in the apps\_dev/scripts/update.R script
    \end{itemize}
  \end{itemize}
\item
  Commit changes
\end{itemize}

\hypertarget{comm}{%
\chapter{Community Members}\label{comm}}

\hypertarget{edit-in-openei}{%
\section{Edit in OpenEI}\label{edit-in-openei}}

\begin{itemize}
\tightlist
\item
  \href{https://marineenergy.app/regs.html}{Regulations}

  \begin{itemize}
  \tightlist
  \item
    \href{https://openei.org/wiki/MarineEnergyApp}{MarineEnergyApp \textbar{} OpenEI}
  \end{itemize}
\end{itemize}

\hypertarget{suggest-in-google-sheets}{%
\section{Suggest in Google Sheets}\label{suggest-in-google-sheets}}

Suggest and then get reviewed by R-technicians for updating.

\begin{itemize}
\tightlist
\item
  \href{https://marineenergy.app/projects.html}{Projects}

  \begin{itemize}
  \tightlist
  \item
    \href{https://docs.google.com/spreadsheets/d/1HC5hXyi2RQSHevnV7rvyk748U5-X3iUw70ewHEfrHm0/edit\#gid=793817660}{projects}
  \end{itemize}
\item
  \href{https://marineenergy.app/ferc.html}{Documents}

  \begin{itemize}
  \tightlist
  \item
    \href{https://docs.google.com/spreadsheets/d/1c9pFSkQyQvLFpyMT4KlBoSFA_wtJ_iNj8YmNX_RZmXc/edit\#gid=951079264}{documents}
  \end{itemize}
\item
  \href{https://shiny.marineenergy.app/report/?nav=Spatial}{Spatial}

  \begin{itemize}
  \tightlist
  \item
    \href{https://docs.google.com/spreadsheets/d/1MMVqPr39R5gAyZdY2iJIkkIdYqgEBJYQeGqDk1z-RKQ/edit\#gid=936111013}{spatial}
  \end{itemize}
\end{itemize}

\hypertarget{applications}{%
\chapter{Applications}\label{applications}}

Some \emph{significant} applications are demonstrated in this chapter.

\hypertarget{example-one}{%
\section{Example one}\label{example-one}}

\hypertarget{example-two}{%
\section{Example two}\label{example-two}}

\hypertarget{final-words}{%
\chapter{Final Words}\label{final-words}}

We have finished a nice book.

  \bibliography{book.bib,packages.bib}

\end{document}
